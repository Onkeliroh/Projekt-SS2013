\documentclass[11pt]{article}
\usepackage[english]{babel}
\usepackage[utf8]{inputenc}
\usepackage{listings}
\usepackage{amsmath}
\usepackage{amsthm}
\usepackage{amssymb} % set symbols i.e. /mathbb{N}
\usepackage{bm}
\usepackage{array}
\usepackage[normalem]{ulem} % underlining
\usepackage{graphicx}
\usepackage{float,rotating}
\usepackage[hang,small,sc]{caption}
\usepackage{subcaption}
\usepackage[final]{pdfpages}
\usepackage{color}
\usepackage[scaled]{helvet}
\renewcommand*\familydefault{\sfdefault} %% Only if the base font of the document is to be sans serif
\usepackage[T1]{fontenc}
\usepackage{hyperref}          %links to chapters
\usepackage{breakurl}
\usepackage{biblatex}

\bibliography{references.bib}

\definecolor{dkgreen}{rgb}{0,0.6,0}
\definecolor{gray}{rgb}{0.5,0.5,0.5}
\definecolor{mauve}{rgb}{0.58,0,0.82}
 
\lstset{
  language=Octave,                % the language of the code
  basicstyle=\footnotesize\ttfamily,           % the size of the fonts that are used for the code
  numbers=left,                   % where to put the line-numbers
  numberstyle=\tiny\color{gray},  % the style that is used for the line-numbers
  stepnumber=2,                   % the step between two line-numbers. If it's 1, each line 
                                  % will be numbered
  numbersep=5pt,                  % how far the line-numbers are from the code
  backgroundcolor=\color{white},      % choose the background color. You must add \usepackage{color}
  showspaces=false,               % show spaces adding particular underscores
  showstringspaces=false,         % underline spaces within strings
  showtabs=false,                 % show tabs within strings adding particular underscores
  frame=single,                   % adds a frame around the code
  rulecolor=\color{black},        % if not set, the frame-color may be changed on line-breaks within not-black text (e.g. comments (green here))
  tabsize=2,                      % sets default tabsize to 2 spaces
  captionpos=b,                   % sets the caption-position to bottom
  breaklines=true,                % sets automatic line breaking
  breakatwhitespace=false,        % sets if automatic breaks should only happen at whitespace
  title=\lstname,                   % show the filename of files included with \lstinputlisting;
                                  % also try caption instead of title
  keywordstyle=\color{blue},          % keyword style
  commentstyle=\color{dkgreen},       % comment style
  stringstyle=\color{mauve},         % string literal style
  escapeinside={\%*}{*)},            % if you want to add LaTeX within your code
  morekeywords={*,...},               % if you want to add more keywords to the set
  columns=fullflexible
}

\parindent=0pt
\begin{document}

\author{}
\title{Kick and Flickable: Communication protocol}
\date{Summer 2013}
\maketitle

\tableofcontents
\newpage

\section{General}
\_ serverAddress = 0;
clients[] = {1, 2, 3}

1 := LED-strip, 2 := Wheel, 3 := Accel
\\

Administration keys:

27 Wheel: "turn\_left" event

28 Wheel: "turn\_right" event

31 Accel: "shake"

45 LED-strip: "move forward"

46 LED-strip: "move backward"

47 LED-strip: "change color"

200 acknowledge packet received

255 test packet received

default unknown adminKey
\\

\_ccPacket.data[0] = receiver;

\_ccPacket.data[1] = sender;
 
\_ccPacket.data[2] = admin \_ key;

\_ccPacket.data[3] = packet number; %This one has not been used yet

\_ccPacket.data[4] = rawRSSI;

\section{Voltage controller}




\section{Server}


\section{Accelerometer}

\section{Ledstrip} 
\end{document}

item What is meant by classification?
We build classes and put a  label into these classes.

\item What features can be used for the purpose?
Color values (RGB) (easy task), position of the image, or SIFT descriptors (high dimension descriptor)

\item Which data reduction technique did we present in the course?
?????


\item What are the basic ideas behind the following classifiers

	\begin{enumerate}
    		\item k-nearest neighbor,
    		 You pick as a class the nearest feature and you group all features into k different classes.
    		 It allows any shape. Is very fast but really sensitive to noise.
    		
    		\item boosting,
    		We have a red and a blue class and we want to find the border between these two classes.
    		This is done by combining different weak classifiers to obtain a complex classifier. 
    		
    		
    		\item Bayesian decision theory and
    		Probabilities
    		We describe similarities with percentages.
    		
    		\item a support vector machine? 
          Here we also have weak classifiers. 
          Simple separation in a high dimensional space    		
    		
    		  		
	\end{enumerate}
\end{enumerate} 
 
% TASK: 

% What does segmentation through watershed transformation mean? Give a short theoretical explanation and segment the given picture \emph{berkeley.jpg} via this method! Please describe your steps and prove your results by providing the segmented images.

