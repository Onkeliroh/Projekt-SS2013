\section{Development}
In order to find ideas of how to implement 'KickandFlickable' devices/interfaces, we first started to learn about the technological components we had at hand, such as the arduino. For example, we made some of the start-up projects published in the lady ada tutorial \cite{lady_ada_arduino_tutorial}.

We learned how to use basic components like LEDs and tilt-sensors. Later, we worked with more complicated components such as the hall effect sensors (See section~\ref{sec:the-wheel}).
Meanwhile, we did experiments with XBees\cite{xbee} and found that its signal range was relatively short for our purposes (the maximum range we obtained was approximately 5 meters). Thus, we decided to start working with Panstamps \cite{panstamp}, which are a compact wireless versions of the Arduino devices.

After exploring different options, we chose the accelerometer for sensing movement in the modules because its output signal is not binary (as the tilt sensor) but continuous and this property gives more flexibility, since not only two but many states can be defined with its data. 

For making the detection between modules possible, we decided to take advantage of the RSSI of the signals produced by the panstamps. Since precision was not a critical factor in our interface, RSSI measurements suited our technical needs. 

To give feedback to the users when they kicked or slightly moved the modules, we decided to use multi color led strands. Light stimulus catches easily user's attention, specially in the night. Moreover, led strands allow the implementation of different color patterns and as a result, different light responses can be given to the users depending on the different ways the move the modules with their feet. 

