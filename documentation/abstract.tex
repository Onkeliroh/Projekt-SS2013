\section{ABSTRACT}
In this work, we explain the implementation of a distributed interface for public spaces where feet play the main role. Users interact with the interface by pushing or kicking the individual modules that constitute the system and each module reacts to the stimulus differently;that is, each one of them expresses a different~\emph{personality}.~\emph{Personalities} are not stored in each module's data but on the  central module's database; as a result, they are not static but configurable.

Each individual module is made up of a sensor and an actuator node; the former gathers information and sends data to the main module while the latter receives data from the main module and reacts to it accordingly.
The central module, on the other hand, is constituted by two entities: a computer with a Java-based server and a server node, both of them interconnected. With the server node as the core, all actuator and sensor nodes establish a wireless network with a star topology, where the information is mainly transferred between a node and the server.  

The Java-Server manages all network nodes depending on the recollected data, and provides an interface to allow users the configuration of different settings in the network. With the interface, users can create their own setups and get a quick and complete overview of the network status.The Java-server determines the personality of each entity. 
 
In the following sections, we describe the technical development of the interface, its final structure and how it works. 

